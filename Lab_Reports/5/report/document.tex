\documentclass[titlepage]{article}

\usepackage[tc]{titlepic}
\usepackage{graphicx}
\usepackage[top=25mm, bottom=25mm, left=27mm, right=27mm]{geometry}
\usepackage{listings}
\usepackage{lstlangarm}
\usepackage{hyperref}




\date{}
\author{Erdal Sidal Dogan\\ \#041702023}
\title{\includegraphics[width=0.6\textwidth]{../../logo_en_color.png}\\ 
\vspace{5em}
EE306 - Microprocessors\\
\vspace{2em}
\textbf{Laboratory Exercise 5 \linebreak Timer
}\\
\vspace{1.5em}
\today}

\begin{document}
	\maketitle
	\section{Task1}
	In this task a LED had been switched on/off with .25 second intervals, which was accurately measured using \textit{A9 Private Timer}.
	\lstinputlisting[language={[ARM]Assembler}, frame=single, basicstyle=\ttfamily, caption= Assembly code of Task-I]{../code/task1.s} 
	\section{Task2}

	\href{https://drive.google.com/file/d/1ZV85GaJMzvtzZx00Zmshr5DKWMjX639F/view?usp=sharing}{Click here for video explanation of Task 2} \\

	This task is the implementation of a clock using 7-segment displays. \textit{A9 Private Timer} is used for accurate timing. The clock counts up to 59.99 seconds then resets. Since the each digit between 0-9 has a unique representation in 7-segment display, these representations are being held in memory location, which the \texttt{SEQUENCE} points the address of.

	\lstinputlisting[language={[ARM]Assembler}, frame=single, basicstyle=\ttfamily, caption= Assembly code of Task-II]{../code/task2.s}
	

\end{document}
